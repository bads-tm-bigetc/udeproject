\documentclass[10pt,a4paper]{article}
\usepackage{html}
\pagenumbering{Roman}

\setlength\parskip{\medskipamount}
\setlength\parindent{0pt}

\newcommand{\myparbox}[2]{
\latexhtml{\parbox{#1}{#2}}{\begin{center}\small#2\end{center}}}

\newcommand{\uwm}{{\sc uwm} }
\newcommand{\ude}{{\sc ude} }

\newenvironment{ttdesc}[1]{
   \begin{list}{}{
          \renewcommand{\makelabel}[1]{\texttt{##1\hfill}}}}{\end{list}}

\begin{document}
\begin{titlepage}
\title{{\Huge \uwm~0.3}\\ user manual}
\author{by Christian Ruppert}
\maketitle
\begin{center}Thanks to everybody who supports us in creating and distributing
this system.\end{center}

\vfill\myparbox{12cm}{
Starting \uwm for the first time you might recognize that it doesn't only look different from other window managers but also behaves not quite the way most of you would first expect such a system to do. This fact alone might be a reason for some people to throw \uwm away and go back to a conventional windowing user interface.\smallskip

Others might start thinking \hfill --- \hfill Some of them might
\textsf{get used to it}}
\thispagestyle{empty}
\pagebreak
\end{titlepage}
\tableofcontents
\pagebreak
\section*{Preface}

\part{A first touch}
\section{\texttt{get used to it}}
\uwm was designed to be easy to use once you're used to it. And although there are many people out there thinking different, getting used to \uwm is not more difficult than getting used to  any other window management system:

How ''intuitive'' is it to press a button with an \textsf{X} painted on to close a window or to double-click somewhere to start a program? -- Well it's exactly as ''intuitive'' as pressing the right mouse button to start programs and pressing the middle button for dragging windows around. Only that people aren't used to this.

In this section we'll take you on a short tutorial tour through the world of \uwm to give you an impression of how to use it. It was written for all those who are curious about \uwm but lost in the differences of the user interface.

\subsection{First steps}
When \uwm starts it presents a blank screen to you. There are no buttons, icons, desktop items etc. to be seen and there will never be during your whole \uwm session (since they would be covered by windows most of the time anyway).

Press the right mouse button and hold it down. A menu titled \textsf{Application menu} appears. Select an application (the preconfigured items might not be available on all systems, but an xterm should be, so move the pointer over the submenu \textsf{xterm} and select the item \textsf{login shell} by releasing the mouse button above the item).

The application's window will appear on the screen. 

Move the pointer somewhere over the window's border and hold down the left mouse button. The famous ude honeycomb will appear. Move the mouse pointer over the hex-icon on the upper left and release the mouse button. Oops... The window just disappeared together with the hex menu. It has been minimized. Some people also call this state of a window iconified. There are no icons for such windows in \uwm however. You might be wondering how to get this damn window back then if it cannot be accessed through an icon. Well, the answer is simple: through a menu!

Press the middle mouse button and hold it down. (Yes, you'll need a three-button-mouse if you want to use \uwm properly. However to get a first touch perhaps your X-server's \textsf{emulate3buttons} will do it). A menu called \textsf{Windows menu} appears and either shows a list of workspaces which will pop out submenus if the pointer moves over them or (in case there is only one workspace) represents the only workspace's submenu itself. The submenus are a list of the windows on the corresponding workspace. Search the only submenu with an entry (which is your program's window of course) and select this entry. The window will deiconify and reappear on the screen.

Now move the pointer somewhere over the window's border again and hold down the middle mouse button. Move the mouse and see what happens: The window is being dragged around. Release the window by releasing the mouse button.

Try out resizing the window using the right mouse button on the window's border.

To finally close the window select the central upper button in the window's hex-menu.

Start another xterm and a second one so that you finally have two windows on the screen now. Move them around to be overlaping. Now press and release the middle mouse button somewhere on the upper window's border without moving the pointer. The window will be lowered under the other one. Reraise the window by pressing the left mouse button and releasing it again somewhere on the border without moving the mouse. (The hex menu will appear as long as the button is pressed -- ignore it.)

Now try around with uwm's keyboard focus handling: Move the mouse over one of the windows. Its border will change its color. The window now has keyboard focus, try it out by typing something. Also type something into the other window.

Close both windows, the first one by typing \texttt{exit} into it, the other one by using the honeycomb's close button.

\subsection{Some things you might want to try out\\
\normalsize(playing the piano)}
Once you're used to uwm's basic ''feel'' as it is described above you might start wondering if this is all or if this piano-like mouse-usage is good for anything except of confusing new users and getting rid of the title-bar. The answer is: It is, and this is one of the things I personally like best about uwm's user interface: The chords.

\subsubsection{Raising/Lowering while moving\\\rm(Tango)}

Open several windows on the screen. Drag a window around using the middle mouse button. You might recognize that this window does not change its stacking position while being dragged. While this effect in most cases is quite useful there might be situations in which you want to raise or lower a window while dragging it and so not only position it two- but threedimensionally.
Press the left mouse button while dragging the window (and release it again, keep the middle button pressed while doing this) -- Whooops: It's risen to the top. Now press the right button and watch your window disappear behind the other ones...

\subsubsection{Multimenu selection\\\rm(Cha Cha)}
Imagine you have several programs put into a subsubsubsubsubmenu of uwm's \textsf{Application Menu} and want to call two of them at a time. I suppose you don't want to call the Menu, work yourself through the submenus to the application, release the button, let the menu disappear and redo all this from the beginning to call the second program. So to make this a little easier simply keep the right button pressed which will keep the menu alive while selecting the programs to be loaded by clicking on the corresponding items with any other mouse button (Amiga users might remember this feature).

Using this method you can also e.g. open several xterms without having to leave and recall the menu in between. Simply click on the xterm item three times in case you want three xterms. Please note that in this context releasing the right mouse button will only load the selected program in case the corresponding item has not been selected by clicking on it directly before releasing the right button.

\part{Command line options}
Currently uwm supports the following command line options:
\begin{ttdesc}{description}
\item[--NoStartScript] will prevent \uwm from executing the StartScript defined in uwmrc (see below).
\item[--NoStopScript] will prevent \uwm from executing the StopScript defined in uwmrc (see below).
\item[--TryHard] conforming to \textsf{icccm 2.0} \uwm has the ability to recognize and replace other (fully) icccm compliant window managers. However the default behaviour is not to replace other window managers since these programs usually will be terminated then which will also terminate the whole X-session in most cases (which shouldn't be done accidentally playing around with \uwm). This option uses the protocol specified by \textsf{icccm 2.0} to make another window manager terminate and pass control to \uwm.
\item[--Hostile] \textsf{icccm 2.0} allows \texttt{X11} resource manager clients to terminate other resource manager clients by deleting their connection to the X-server in case they do not react to the protocol used by \texttt{uwm --TryHard}. \texttt{uwm --Hostile} uses this in addition to the \texttt{uwm --TryHard} protocol. However there might be some really ugly side effects killing a window managers connection to the X-server so be careful with this option.
\item[--StayAlive] tells \uwm exactly not to react on protocol requests like the ones used by \texttt{uwm --TryHarder}. However \texttt{uwm --Hostile} is able to replace a \texttt{uwm --StayAlive}.
\item[--help] displays a brief help screen describing \uwm's command line options.
\end{ttdesc}

\part{The graphical user interface}
\begin{center} --- will be added later --- \end{center}
\section{Menus}
\begin{center} --- will be added later --- \end{center}
\section{Windows}
\begin{center} --- will be added later --- \end{center}
\section{The Keyboard}
Currently the following keyboard shortcuts are implemented. This is not configurable yet (but will hopefully be in future releases):

\begin{tabular}{p{4cm}p{8cm}}
\texttt{CTRL+ALT+LEFT\_ARROW} & go to previous workspace\\
\texttt{CTRL+ALT+RIGHT\_ARROW} & go to next workspace\\
\texttt{CTRL+ALT+UP\_ARROW} & activate next window\\
\texttt{CTRL+ALT+DOWN\_ARROW} & activate previous window\\
\texttt{CTRL+ALT+DOWN\_ARROW} & activate previous window\\
\texttt{CTRL+ALT+PG\_UP} & raise active window\\
\texttt{CTRL+ALT+PG\_DOWN} & lower active window\\
\texttt{CTRL+ALT+}mouse button inside any window & act as if mouse button was pressed on the corresponding window's border
\end{tabular}

\begin{center} --- will be extended later --- \end{center}

\part{Configuration}\label{Configuration}
When \uwm starts it first searches and reads its central configuration file,
\textrm{uwmrc}. Any further configuration files to be read by \uwm have to be
defined in this file. If no \textsf{uwmrc} is found, \uwm will use
default values. However the values assumed might not be correct in all cases
which sometimes is a reason for uwm to quit (e.g. in case no hex icons can be
found).

Configuration files are searched using the following pattern: \uwm first looks
into the directory \texttt{\$HOME/.ude/config}. If the file is not found there
it checks the directory \texttt{\$UDEdir/config} or, if the \texttt{\$UDEdir}
environment variable is not set, the global \ude configuration directory which
depends on your installation but often is something like
\texttt{/usr/local/share/ude/config}. The \ude
default installation directory can be changed at compile/configuration time.
Take a look at the \texttt{INSTALL}-Readme-File for details about this. If the
configuration file is still not found, \uwm takes the filename as it is.

If a C preprocessor is found on the system, configuration files are passed
through this preprocessor before being parsed.

Environment variables can be used on arbitrary positions in configuration
files. They are dereferenced after the C preprocessor but before the file is
parsed by \uwm. The only accepted notation for environment variables is
\texttt{\$\{VARIABLE\}}, which is replaced by the value of \texttt{VARIABLE}.
In particular, \texttt{\$VARIABLE} does \textbf{not} work.

\section{Configuration File Syntax}
Configuration files contain a list of expressions, each defining a
\begin{itemize}
\item External Data to Include
\item Global Setting
\item Workspace specific Setting
\item Menu
\item Event
\end{itemize}

\subsection{Including external data}
\subsubsection{Including files}
There are two ways of including files at an arbitrary position in uwm
configuration files: The C preprocessor's \#include and the \texttt{FILE}
statement.

The C preprocessor's include should only be used to include files containing
macro definitions for the preprocessor. In all other cases, use the
\texttt{FILE} statement to include Files after the preprocessing stage:

\begin{verbatim}
FILE "/path/to/file";
\end{verbatim}

Each file included using the \texttt{FILE} statement is also passed through
the C preprocessor and files are searched for in directories as described
above.

\subsubsection{Including the Output of an arbitrary Shell Command}
\uwm permits to include the output of an arbitrary \texttt{/bin/sh}
command into its configuration files using the \texttt{PIPE} statement:

\begin{verbatim}
PIPE "shell command";
\end{verbatim}

\uwm executes shell command and replaces the \texttt{PIPE} statement by its
stdout. This can be used to include all files in a directory or to do other
very cool stuff with it, e.g. writing a short shell script that creates a menu
to display all .dvi files in a documentation directory. Please note that the
shell command is evaluated exactly once when uwm reads its configuration and
never again until uwm is restarted.

\subsection{Global Settings}

\uwm's overall layout and behaviour is controlled by global settings. They are
set using the syntax

\begin{verbatim}
GLOBAL Setting = Value;
\end{verbatim}

or

\begin{verbatim}
GLOBAL {
  Setting1 = Value1;
  Setting2 = Value2;
  ...
}
\end{verbatim}

Currently, the following global settings are implemented in \uwm:
\subsubsection{Geometry Settings}
\begin{ttdesc}{description}
\item[BorderWidth (INT, 10)] The width of window borders in pixels.
\item[TransientBorderWidth (INT, 3)] The widht of so-called transient window's
borders in pixels. Most applications mark requesters and other temporary
dialogue windows as transient.
\item[TitleHeight (INT, 0)] The number of pixels the northern border is wider
than the other ones.
\item[FrameBevelWidth (INT, 2)] The width of the 3D-effect used to draw window
borders in pixels.
\item[FrameFlags (INT, 39)] A sum of the following values that defines the
layout of window borders:\\[\smallskipamount]
\begin{tabular}{rlp{7cm}}
$1$  & \texttt{GROOVE} & draw the groove on window borders if there's enough
space.\\
$2$ & \texttt{CENTER\_TITLE} & display titles in the center of the top border
instead of the northeastern corner.\\
$4$  & \texttt{BLACK\_LINE} & draw a black separation line along the inside of
window borders.\\
$8$ & \texttt{DODGY\_TITLE} & hide active window's title when hit by the mouse
pointer.\\
$16$  & \texttt{ACTOVE\_TITLE} & display active window's title.\\
$32$  & \texttt{INACTIVE\_TITLE} & display inactive windows' titles.\\
\end{tabular}\\

\item[HexPath (STRING)] The directory in which \uwm searches for the hex icon
set to be used. The icons must be in \texttt{.xpm} format and have the
following names and meanings:\\[\smallskipamount]
\begin{tabular}{llp{5cm}}
Normal State & Selected State & Meaning\\
\texttt{autorise.xpm} & \texttt{autorises.xpm} & autorise or resize the
window\\
\texttt{back.xpm} & \texttt{backs.xpm} & lower the window \\
\texttt{close.xpm} & \texttt{closew.xpm} & close the window \\
\texttt{iconify.xpm} & \texttt{iconifys.xpm}& iconify/minimize the window\\
\texttt{kill.xpm} & \texttt{kills.xpm} & shut down the application's
connection to the X-server \\
\texttt{menu.xpm} & \texttt{menus.xpm} & open the window menu\\
\texttt{really.xpm} & \texttt{reallys.xpm} & security button that X-Server
connection gets killed by accident \\
\end{tabular}\\
If a button's shape is to be different from the hexagonal default shape, it
must be defined in the xpm file. The button's coordinates relative to the
center must be defined as hotspot coordinates of the xpm.

\end{ttdesc}

\subsubsection{Layout Settings}
\begin{ttdesc}{description}
\item[LayoutFlags (INT, 0)] A sum of the following values that defines \ude's
general layout:\\[\smallskipamount]
\begin{tabular}{rlp{7cm}}
$1$  & \texttt{SUBMENU\_TITLES} & Display redundant submenu titles in
autogenerated menus.\\
\end{tabular}\\

\item[BevelWidth (INT, 2)] The width of the 3D-effects of menus, buttons etc.
in pixels.
\item[MenuXOffset (INT, 2)] The distance of a menu's text from the left and
right borders in pixels.
\item[MenuYOffset (INT, 2)] The distance of a menu's text from the upper and
lower borders in pixels.
\end{ttdesc}

\subsubsection{Fonts}
\begin{ttdesc}{description}
\item[TitleFont (FONT, "fixed")] The font used for window titles.
\item[Font (FONT, "fixed")] The font used for menus and other GUI elements.
\item[MonoFont (FONT, "fixed")] The monospace character set for text editors
etc.
\item[HighlightFont (FONT, "fixed")] The font used for highlighted items on
GUI elements.
\item[InactiveFont (FONT, "fixed")] The font used for inactive items on GUI
elements.
\end{ttdesc}

\subsubsection{Behaviour Settings}
\begin{ttdesc}{description}
\item[PlacementStrategy (INT, 5)] The placement strategy to be used. The
following values can be used:\\[\smallskipamount]
\begin{tabular}{lp{10cm}}
0 & no placement strategy\\
1 & gradient-placement (automatic placement)\\
2 & agressive gradient-placement (place \textbf{all} windows automatically)\\
3 & gradient-placement (automatic placement)\\
4 & agressive interactive placement (place \textbf{all} windows semi-automatically) \\
5 & interactive placement (semi-automatic placement)\\
6 & agressive user placement (place \textbf{all} windows manually)\\
7 & user placement (manual placement)
\end{tabular}

\item[PlacementThreshold (INT, 0)] The overlapping value in pixels from which
on you want to place your windows manually in interactive placement stratrgy.
This is useless in other placement strategies. In most other WMs $0$ is used
here without any comments or a way to change. If you want this option to make
sense your values shouldn't be too small (I tried out $100000$ to be quite a
good value at a screen-size of $1200\times1024$).
\item[OpaqueMoveSize (INT, 0)] specifies the size in pixels from which on
windows are no longer moved opaquely but transparently. A value of 0, which is
also the default, means move always opaque, any other value means move
transparent from that size on. Values greater than 0 might be useful when you
are using certain applications on slower machines. You should try out your
favourite value or if e.g. transparent movement for all windows works better
on your machine with your frequently used applications.
\item[MaxWinWidth (INT, 0)] The maximum window width allowed. This is useful
on displays with low resolutions. Unfortunately there might be some problems
with applications not regarding the most basic X11 specifications. 0 means
unlimited width.
\item[MaxWinHeight (INT, 0)] The maximum window height allowed. This is useful
on displays with low resolutions. Unfortunately there might be some problems
with applications not regarding the most basic X11 specifications. 0 means
unlimited height.
\item[WarpPointerToNewWinH (FLOAT, -1)] This allows you to make uwm warp the
pointer to any newly mapped window. Any value between 0 and 100 defines the
X-position in the window (in percent) the pointer is warped to. A value of -1
means don't warp the pointer and a value of -2 means warp the pointer to the
upper left corner of the window's border. Only takes effect if
\textsf{WarpPointerToNewWinV} is set between 0 and 100.
\item[WarpPointerToNewWinV (FLOAT, -1)] This allows you to make uwm warp the
pointer to any newly mapped window. Any value between 0 and 100 defines the
Y-position in the window (in percent) the pointer is warped to. A value of -1
means don't warp the pointer and a value of -2 means warp the pointer to the
upper left corner of the window's border. Only takes effect if
\texttt{WarpPointerToNewWinH} is set between 0 and 100.
\item[SnapDistance (INT, 10) The distance (in pixels)] from which a window snaps to another window's or the screen's border when being moved.
\item[BehaviourFlags (INT, 0)] A sum of the following values that defines the
general behaviour of \ude:\\[\smallskipamount]
\begin{tabular}{rlp{7cm}}
$1$  & \texttt{TRANSIENT\_MENUS} & If this is set menus disappear as soon as
the mouse button is released. If it's not set, menus called with a short mouse
click stay alive until an item is selected.\\
$2$  & \texttt{ALL\_CLICKS} & do not ignore mouse events passed on to uwm by
some client windows (e.g. xosview can be moved easily clicking somewhere in
the window using this option).\\
\end{tabular}\\
\end{ttdesc}

\subsubsection{Miscellaneous Settings}
\begin{ttdesc}{description}
\item[StartScript (STRING)] A shell command that is executed when \uwm is
started after reading the configuration files.
\item[StopScript (STRING)] A shell command that is executed before \uwm
terminates.
\item[ResourceFile (STRING)] \uwm can read a file with the format described in
section \ref{urdb} to set workspace specific application colors etc.
\end{ttdesc}

\subsection{Workspace specific Settings}
For several options such as desktop colors, uwm allows workspace specific
settings:

\begin{verbatim}
WORKSPACE number Setting = Value;
\end{verbatim}

or

\begin{verbatim}
WORKSPACE number {
  Setting1 = Value1;
  Setting2 = Value2;
  ...
}
\end{verbatim}

Currently, the following workspace specific settings are implemented in \uwm:

\begin{ttdesc}{description}
\item[Name (STRING)] The workspace's name.
\item[Wallpaper (PIXMAP)] The filename of an image used as screen background
for the workspace. Must be either in JPEG or xpm format.
\item[ScreenColor (COLOR, "grey30")] The workspace's screen background color.
\item[InactiveColor (COLOR, "grey30")] The color of inactive windows' borders.
\item[InactiveShadow (COLOR, "grey10")] The color of the shadows on inactive
windows' borders.
\item[InactiveLight (COLOR, "grey50")] The color of the highlights on inactive
windows' borders.
\item[InactiveTitle (COLOR, "white")] The color of inactive windows' titles.
\item[ActiveColor (COLOR, "grey70")] The color of active windows' borders.
\item[ActiveShadow (COLOR, "grey50")] The color of the shadows on active
windows' borders.
\item[ActiveLight (COLOR, "grey90")] The color of the highlights on active
windows' borders.
\item[ActiveTitle (COLOR, "black")] The color of inactive windows' titles.
\item[BackgroundColor (COLOR, "grey30")] The default background color.
\item[BackgroundShadow (COLOR, "grey10")] The color for shadows on default
backgrounds.
\item[BackgroundLight (COLOR, "grey50")] The color for highlights on default
backgrounds.
\item[ForegroundColor (COLOR, "white")] The default foreground color.
\item[InactiveForeground (COLOR, "grey70")] The color for inactive
foregrounds.
\item[InactiveBackground (COLOR, "grey20")] The color for inactive
backgrounds.
\item[HighlightedForeground (COLOR, "white")] The color for highlighted
foregrounds.
\item[HighlightedBackground (COLOR, "grey50")] The color for highlighted
backgrounds.
\item[TextForeground (COLOR, "white")] The foreground color for text boxes.
\item[TextBackground (COLOR, "black")] The background color for text boxes.
\end{ttdesc}

\section*{TODO: DATATYPES}

\subsection{Menus}
\subsection{Event}

In this description the following data types are being used as arguments for the options:
\begin{ttdesc}{description}
\item[<nr>] is an integer number with the range specified in the option's description.
\item[<string>] represents a usual text line. it may contain any desired characters, whitespace etc. and is terminated by a linebreak.
\item[<font>] is an X11 font definition string. The most easy way to get such a string is to paste it directly from xfontsel into the file.
\item[<filename>] is the name of a file. The file is searched in the way described above and in most cases passed through the c preprocessor.
\item[<col>] represents an X11 color definition string. For the exact format of these strings please take a look at the man page of \texttt{XQueryColor}. All colors can be set for any workspace seperately.
\item[<triple>] represents a set of three semicolon seperated integers.
\item[<float>] represents a floating point number. Please note that the decimal expected seperation character may differ with the internationalized version of (g)libc with different \texttt{LANG}-environments set (e.g.\texttt{.} as default but \texttt{,} for \texttt{LANG=de}). Admins of multilingual systems say thanx to the big internationalisationers of libc for this feature.
\item[\{X|Y|Z\}] means one out of \texttt{X}, \texttt{Y} or \texttt{z}.
\end{ttdesc}


\section{Menu definition files}
A menu definition file is a hierarchical file made up of the following commands:
\begin{ttdesc}{description}
\item[SUBMENU "<name>" \{\textsf{commands to build submenu}\}] will create a submenu named \texttt{<name>} with the items created by the commands inside the braces.
\item[ITEM "<name>":"<command>";] will create an item on the corresponding position named \texttt{<name>} which will lead to the execution of \texttt{command} if selected. The item is not created in case there already exists an item with the same \texttt{<name>} in the same submenu.
\item[LINE;] will add a seperation line to the corresponding position. Several \texttt{LINE}s with nothing else in between will be truncatd to a single seperator.
\item[FILE "<filename>";] will process the named file as if its contents were in the position of the \texttt{FILE} command. The file is searched for in the way described above and passed through the preprocessor.
\item[PIPE "<command>";] will call \texttt{<command>} and process its standard output as if it was in the position of the \texttt{PIPE} command. The commands output is not passed through the preprocessor.
\end{ttdesc}


\section{urdb files}
\label{urdb}
urdb-files have the same format as xrdb files (see xrdb documentation for details) except that there are some additional macros defined which are replaced by uwm workspace dependent. These macros are:
\begin{ttdesc}{description}
\item[@BACKGROUND@] represents the workspace's background color set with \texttt{BackColor} in uwmrc.
\item[@LIGHTCOLOR@] represents the workspace's light bevel color for \texttt{BackColor}.
\item[@SHADOWCOLOR@] represents the workspace's shadow bevel color for \texttt{BackColor}.
\item[@STANDARDTEXT@] represents the workspace's standard text color (\texttt{TextColor}).
\item[@INACTIVETEXT@] represents the workspace's color for inactive text (\texttt{InactiveText})
\item[@HIGHLIGHTEDTEXT@] represents the workspace's color for highlighted text (\texttt{HighlightedText})
\item[@HIGHLIGHTEDBGR@] represents the workspace's background color for highlighted text (\texttt{HighlightedBgr})
\item[@TEXTCOLOR@] represents the workspace's text color for text windows (\texttt{TextColor})
\item[@TEXTBGR@] represents the workspace's background color for text windows (\texttt{TextBgr})
\item[@BEVELWIDTH@] the standard bevel width (\texttt{BevelWidth}).
\item[@FLAGS@] 1 if transient menus are activated, if not 0.
\item[@STANDARDFONT@] the standard font (\texttt{MenuFont}).
\item[@INACTIVEFONT@] the font for inactive buttons etc (\texttt{InactiveFont})
\item[@HIGHLIGHTFONT@] the font used for highlighted text (\texttt{HighlightFont})
\item[@TEXTFONT@] the font used for text windows (\texttt{TextFont})
\end{ttdesc}
\end{document}
